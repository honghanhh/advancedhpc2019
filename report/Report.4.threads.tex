\documentclass{article}
\usepackage[utf8]{inputenc}
\usepackage{graphicx}
\usepackage{tikz} 
 \usetikzlibrary{arrows,automata,positioning,petri}
 
\title{Report on Labwork 4}
\author{TRAN Thi Hong Hanh}

\begin{document}

\maketitle
\section{Explain how you improve the labwork?}
    \begin{itemize}
    \item Redefine the grayscaling kernel
    \begin{verbatim}
    __global__ void grayscale2D(uchar3 *input, uchar3 *output, 
                                        int width, int height) {
        int globalIdX = threadIdx.x + blockIdx.x * blockDim.x;
        if (globalIdX >= width) 
            return;
        int globalIdY = threadIdx.y + blockIdx.y * blockDim.y;
        if (globalIdY >= height) 
            return;
        int tid = globalIdY * width + globalIdX;
    
        output[tid].x = (input[tid].x + input[tid].y +input[tid].z) / 3;
        output[tid].z = output[tid].y = output[tid].x;
    }
    \end{verbatim}
    \item Redefine blockSize and gridSize
    \end{itemize}
    \begin{verbatim}
    dim3 blockSize = dim3(32, 32);
    dim3 gridSize = dim3((inputImage->width + blockSize.x -1) / blockSize.x,
                         (inputImage->height + blockSize.y -1) / blockSize.y);
    grayscale2D<<<gridSize, blockSize>>>(devInput, devOutput,
                                        inputImage->width, inputImage -> height);
    \end{verbatim}
    
\section{Try experimenting with different 2D block size values?}
    \begin{table}[]
    \centering
    \begin{tabular}{|l|l|l|l|l|}
    \hline
    \multicolumn{1}{|c|}{\textbf{Block size}} & 64 & 128 & 256 & \multicolumn{1}{c|}{512} \\ \hline
    \textbf{Time elapsed (ms)} & 11.5 & 10.4 & 10.7 & 11.0\\ \hline
    \end{tabular}
    \end{table}
    
\section{Compare speedup with previous 1D grid}
\noindent
It is faster compared to previous 1D grid.

\end{document}

